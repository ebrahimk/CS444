\documentclass[10pt,onecolumn,draftclsnofoot]{IEEEtran} %may need comma after
\usepackage{times}
\usepackage{graphicx}
\usepackage{array}
\usepackage{float}
\usepackage{geometry}
\usepackage{titling}
\usepackage{hyperref}
\usepackage{setspace}
\usepackage{listings}
\usepackage{color}
\definecolor{mygreen}{rgb}{0,0.6,0}
\definecolor{mygray}{rgb}{0.5,0.5,0.5}
\definecolor{mymauve}{rgb}{0.58,0,0.82}
\lstset{
        basicstyle=\footnotesize,
        commentstyle=\color{mygreen},
        frame=single,
        keywordstyle=\color{blue},
        numberstyle=\tiny\color{mygray},
        numbers=left,                    % where to put the line-numbers; possible values are (none, left, right)
        numbersep=5pt,
        language=bash,
        rulecolor=\color{black},
        }
\setlength{\parindent}{1cm}
\newgeometry{left=1.905cm, right=1.905cm}
%this is a comment
\title{ Project Three, CS444, Spring 2018, Homework Group 10}
\author{Kamron Ebrahimi, Maximillian Schmidt \& Brendan Byers \\ ebrahimk, schmidtm, byersbr }
\date{\today}

\begin{document}

\begin{titlingpage}
\maketitle
\begin{abstract}
\begin{singlespace}
This report details the process of setting up and configuring a Linux block device module and the specific aspect of the Linux kernel \textit{crypto} api. In this project the team produced block device driver which behaved like a RAM  disk. The disk would encrypt written data upon successful writes and would decrypt said data successfull reads. All encryption and decryption was performed using the Linux Kernel \textit{crypto} api, in particular the kernel \textit{AES} cryptography algorithm. In addition to producing this block driver the team also learned how to import and configure Kernel modules a critical concept for Linux Kernel tuning. 
\end{singlespace}
\end{abstract}
\end{titlingpage}


\tableofcontents

\newpage
\begin{singlespace}
\section{\bf  Design and Implementation}

  \normalfont \indent The team started this assignment by first researching Linux Kernel modules and finding examples of simple block device drivers. After settling on the \textit{sbd} template the team began researching how each function within the \textit{sbd} implementation worked. It was determined that the \textit{sbd\_init()} registered tyhe block driver with the kernel by providing a \textit{major num} to the driver. This function also allocated the proper amount of space for the block driver. The team determined that this function would also neeed to contain all initializaiton of structures used to implement the cryptographic requirements for this assignment.

  \normalfont \indent The \textit{sbd\_request} simply managed the servicing of incoming requests to the driver. From within this function is a call to \textit{sbd\_tranfer} which data describing the incoming request is actually transfrered to the output queue for servicing. In the case of this assignment the team determined that the encrytion of all data contained within an incoming request was to be encryted before being dispatched and serviced, i.e. pyhsically written to the RAM disk. 


   \normalfont \indent With a rough idea of how the algorithm would be implemented in mind the team set about understanding the Kernel \textit{crypto} api. The \textit{crypto} api suffers from remarkably poor, sparse documentation. From class lectures the team knew that utilizing the \textit{AES} encrytion algorithm would be the most straightforward approach to completing the assignment. Thus instead of trying to dig through pages of poor \textit{crypto} documentation, the team tried to find reference examples in which the \textit{AES} algorithm is used. This produced fruitful results and after some exhaustive testing the team was able to figure out how to implement the \textit{AES} algorithm.  

   \normalfont \indent The final step to completing the algorithm was to tie all of the ends together and figue out how to actually import the module and mount the ext2 file system. This was accomplished with a variety of commands such as \textit{mkfs.ext2}, \textit{mount}, and \textit{insmod}. With these commands the kernel module's \textit{.ko} object could be transfered within a running instance of the virtual machine via \textit{scp} and inserted into the kernel. The file system could then be produced and mounted within the VM and reading and writing to the file system could ensue. 

\section{\bf Questions}

        \textbf{What do you think the main point of this assignment is?}\\

                \normalfont \indent The purpose of this assignment was to familiarize the team with writing and configuring kernel block device modules. In particular this project offered insight into some of the memory management submodules within the Linux Kernel. The assignment also allowed the team to gain some first hand experience using poorly document apis. A great deal of kernel source code is poorly documented and this project stregthened the teams ability to utilize outside resources. 
        
	%discuss testing and python scripts used to grab and parse data 
        \textbf{How did you ensure your solution was correct?}\\

                \normalfont \indent

        \textbf{What did you learn?}\\
        \normalfont \indent This assignment was a perfect introduction to writing modules in the Linux Kernel. The project perfectly illustrated how Linux goes about providing a clean and easy interface for importing customized modules. Additionally this assignment showed the team just how poorly documented some aspects of Linux Kernel are and drove home the point that often information is not just readily avaliable and prepackaged for consumption but must be extrapilated from a variety of resources. The team really had to dig through examples and test alot of iterations in order to get the \textit{crypto} algorithm working properly.

        \textbf{How should the TA evaluate your work?}\\
        
	\normalfont \indent


	\begin{enumerate}
                \item 
                \item 
                \item 
		\item
		\item
		\item
		\item
		\item
		\item
		\item
        \end{enumerate}

\newpage
\section{\bf Work Log}

                \begin{tabular}{l l l l}\textbf{Link} & \textbf{Date} & \textbf{Author} & \textbf{Description}\\\hline
\href{https://github.com/ebrahimk/CS444/commit/e10cf2a5be3cb7528d156eb991bb4ee94c5ac0f3}{e10cf2a} & Tue Apr 10 22:42:32 2018 -0700 & ebrahimk & added Makefile and tex template\\\hline
\href{https://github.com/ebrahimk/CS444/commit/399064e4a3c386c1aed3ebf025f2e964411049fb}{399064e} & Tue Apr 10 22:45:33 2018 -0700 & ebrahimk & made project1 folder\\\hline
\href{https://github.com/ebrahimk/CS444/commit/7e41f8e334d9b67756aa7fb715773a4360e83a3b}{7e41f8e} & Tue Apr 10 22:51:18 2018 -0700 & ebrahimk & Added another tex template\\\hline
\href{https://github.com/ebrahimk/CS444/commit/90bed413ed27774ee7f3d84703ea90d9f32e4484}{90bed41} & Wed Apr 11 13:41:22 2018 -0700 & cascadeth & Added scripts for easy setup of the VM for hw1\\\hline
\href{https://github.com/ebrahimk/CS444/commit/225209cf1bb27caaab8d148d66ad2cebcc347828}{225209c} & Wed Apr 11 17:35:33 2018 -0700 & ebrahimk & Progress on tex report\\\hline
\href{https://github.com/ebrahimk/CS444/commit/a22de85663bb053c8335c5e3592ea39b65c292a0}{a22de85} & Wed Apr 11 17:43:53 2018 -0700 & ebrahimk & IEEEtran styling added\\\hline
\href{https://github.com/ebrahimk/CS444/commit/466bf6d68d7bb5055f687bad89014f47d1a98575}{466bf6d} & Wed Apr 11 20:22:08 2018 -0700 & ebrahimk & Git log added for reports\\\hline
\href{https://github.com/ebrahimk/CS444/commit/bfdf3b3f6ce5e223fe99ec5bf17eb73c64853601}{bfdf3b3} & Wed Apr 11 20:34:19 2018 -0700 & ebrahimk & remove log\\\hline
\href{https://github.com/ebrahimk/CS444/commit/604b920fe6c569ed6a59f81d457f5c07590b2661}{604b920} & Thu Apr 12 15:22:05 2018 -0700 & cascadeth & Updated script with colors and clarifications; updated readme with clarifications\\\hline
\href{https://github.com/ebrahimk/CS444/commit/aed5444563d69dd80a0e0dc05e9659e82afc980a}{aed5444} & Thu Apr 12 16:24:00 2018 -0700 & cascadeth & updated gitignore\\\hline
\href{https://github.com/ebrahimk/CS444/commit/c18877d67d024ff8a33b14138f21d754cd89d805}{c18877d} & Thu Apr 12 16:32:36 2018 -0700 & cascadeth & removed .DS_Store files; now included in gitignore\\\hline
\href{https://github.com/ebrahimk/CS444/commit/7ee3a3b2feefb1a785d704251be9c1867c7bd3cb}{7ee3a3b} & Thu Apr 12 17:17:11 2018 -0700 & ebrahimk & Added Gitlog script\\\hline
\href{https://github.com/ebrahimk/CS444/commit/603e40262677cec416eb147b7e31a03c597cfce1}{603e402} & Thu Apr 12 18:45:58 2018 -0700 & bl3rg & First concurrency assignment, should be complete\\\hline
\href{https://github.com/ebrahimk/CS444/commit/7d5d6439601af7d72474fcf711d44191da20b207}{7d5d643} & Thu Apr 12 18:59:48 2018 -0700 & bl3rg & Completed first concurrency assignment\\\hline
\href{https://github.com/ebrahimk/CS444/commit/b6b9eb246d8520fe1435c3c33019cc1dbc70b537}{b6b9eb2} & Mon Apr 16 21:37:03 2018 -0700 & ebrahimk & Reorganized files according to assignment-1\\\hline
\href{https://github.com/ebrahimk/CS444/commit/82ecafd78c700dbc91097ddf33fdac2f1f6d9419}{82ecafd} & Fri May 4 13:14:17 2018 -0700 & Kamron & added my github name\\\hline
\href{https://github.com/ebrahimk/CS444/commit/dff1676205e257abd96e343da4013e50ae01e203}{dff1676} & Fri May 4 13:18:30 2018 -0700 & Kamron & adding project1 file updates\\\hline
\href{https://github.com/ebrahimk/CS444/commit/d5b2af8a3f44f2fba74528489a524bba680763b2}{d5b2af8} & Fri May 4 13:20:52 2018 -0700 & Kamron & assignment 2\\\hline
\href{https://github.com/ebrahimk/CS444/commit/f2110844dc7d778d3f6d29a7a0e235f2ca69d4df}{f211084} & Fri May 4 13:44:34 2018 -0700 & Kamron & Merge branch 'master' of https://github.com/ebrahimk/CS444\\\hline
\href{https://github.com/ebrahimk/CS444/commit/11ee94cb0b8000f4317934ec3ab14919fc8e8495}{11ee94c} & Fri May 4 13:49:32 2018 -0700 & Kamron &  reorganizing\\\hline
\href{https://github.com/ebrahimk/CS444/commit/f274fb3d1cd919b019742c322e3ea8d98f55cca7}{f274fb3} & Fri May 4 13:50:51 2018 -0700 & Kamron & reorganizing\\\hline
\href{https://github.com/ebrahimk/CS444/commit/ee2720ac8cdb00dd6e09203976606ff0e98010b6}{ee2720a} & Fri May 4 13:51:48 2018 -0700 & Kamron & reorganizing\\\hline
\href{https://github.com/ebrahimk/CS444/commit/86650ed58df77c2dc69b1aed453878ec360b6526}{86650ed} & Fri May 4 14:36:56 2018 -0700 & Kamron & added python test script modified qemu script\\\hline
\href{https://github.com/ebrahimk/CS444/commit/68bb5f47a651687c5de0a882a70d1f35a0064165}{68bb5f4} & Fri May 4 17:57:10 2018 -0700 & Kamron & generated python script to grab sector data for plot\\\hline
\href{https://github.com/ebrahimk/CS444/commit/048e41943ebf06341dcfd826548f5ed68a435277}{048e419} & Fri May 4 23:51:38 2018 -0700 & Kamron & adding look implementation\\\hline
\href{https://github.com/ebrahimk/CS444/commit/c80f0e6bb8a253f2bc8328b9fe55efcfe095da42}{c80f0e6} & Sat May 5 01:48:06 2018 -0700 & Kamron & updated scheduler\\\hline
\href{https://github.com/ebrahimk/CS444/commit/a231bfbe9e4e5321231b75e5e036dccd86509853}{a231bfb} & Sat May 5 14:08:16 2018 -0700 & Kamron & added merging functionality to CLOOK algorithm as well as a merge testing python script\\\hline\end{tabular}


\newpage


\end{singlespace}
\restoregeometry
\end{document}


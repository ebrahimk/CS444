\documentclass[10pt,onecolumn,draftclsnofoot]{IEEEtran} %may need comma after
\usepackage{times}
\usepackage{graphicx}
\usepackage{array}
\usepackage{float}
\usepackage{geometry}
\usepackage{titling}
\usepackage{hyperref}
\usepackage{setspace}
\usepackage{listings}
\usepackage{color}
\definecolor{mygreen}{rgb}{0,0.6,0}
\definecolor{mygray}{rgb}{0.5,0.5,0.5}
\definecolor{mymauve}{rgb}{0.58,0,0.82}
\lstset{
        basicstyle=\footnotesize,
        commentstyle=\color{mygreen},
        frame=single,
        keywordstyle=\color{blue},
        numberstyle=\tiny\color{mygray},
        numbers=left,                    % where to put the line-numbers; possible values are (none, left, right)
        numbersep=5pt,
        language=bash,
        rulecolor=\color{black},
        }
\setlength{\parindent}{1cm}
\newgeometry{left=1.905cm, right=1.905cm}
%this is a comment
\title{ Homework 4, CS444, Spring 2018, Homework Group 10}
\author{Kamron Ebrahimi, Maximillian Schmidt \& Brendan Byers \\ ebrahimk, schmidtm, byersbr }
\date{\today}

\begin{document}

\begin{titlingpage}
\maketitle
\begin{abstract}
\begin{singlespace}
The report focuses on the creation of a best-fit allocation algorithm and how effeciently it allocates memory compared the the SLOB allocator, which operates using a first-fit algorithm.. 
\end{singlespace}
\end{abstract}
\end{titlingpage}


\tableofcontents

\newpage
\begin{singlespace}
\section{\bf  Design and Implementation Prior To Execution}

  \normalfont \indent The team started this assignment by first researching the allocation algorithms present in the Linux Kernel. To handle memory allocations that are smaller than a page, one of three allocation algorithms are used: SLAB, SLUB, and SLOB.  SLAB and SLUB are more complex allocation algorithms that are often much more efficient than SLOB.  SLAB allocation works be pre-allocating memory in caches so that when a request to allocate memory is received, it can immediately fill that request.  Unlike SLOB, SLAB eliminates fragmentation due to it allocating an entire "slab" at once.  SLUB was created as a response to the SLAB allocator, which requires a lot of overhead to operate.  SLUB creates a \textit{slab} similar to the SLAB algorithm, but it will manipulate these slabs as needed, such as modifying queue sizes, merging slabs, and partial slabs.  Also, instead of storing metadata about the slab at the beginning of the block, SLUB keeps the metadata in a page struct, meaning that the data stored in the slab will be aligned to the block and fit within a 4k block without needing extra space. 
  \normalfont \indent For this assignment we will be comparing our best-fit allocation scheme to a SLOB implementation.  SLOB is a first-fit algorithm, meaning that when served with a request it will reference a list of free blocks, allocating to the first one that is large enough for the request.  This can result in a large amount of fragmentation because the first block that fits may not be the optimal block. The best fit algorithm that we will be creating will try and place blocks in the best spot, so there is less fragmentation  
	\normalfont \indent The team started their design by first analyzing the \textit{slob.c}. It is apparent that the \textit{slob\_page\_alloc()} is the function in which the linked list of blocks is traversed to find the first fit and the best fit block is configured and returned. Manipulating this function will be the first objective of the team. To implement a best-fit algorithm the team plans on taking a simple approach by maintaining several local variables which will keep track of the best fit \textit{slob\_t} object in the list as well as any other pertinent variables used to actually allocate the memory within the best fit block. Judging by the source slob code the actual object configuration takes place after line 232 and utilizes a number of confusing variables such as \textit{delta} and \textit{aligned} and some more straightforward variables such as \textit{prev} and \textit{avail}. To reuse as much code as possible the team plans to save separate instances of all of these variables which reference the current best fit \textit{slob\_t} object. While traversing the linked list of blocks if a new best fit is found then the second instance of these variables will all be updated. In this fashion the team can compartmentalize the work done by this function. The first half of the function will traverse the linked list of blocks, upon finding a block size that is a better fit for the memory allocation request a pointer to the new best fit block will be updated as well as a pointer to the \textit{prev} node of the best fit block, the \textit{delta} value of the best fit block and a pointer to the \textit{aligned} node of the best fit block. The second part of the function will perform the actual configuration of the \textit{slob\_t} object by reusing the exact same code used in the \textit{first-fit} slob approach. The key difference is that the segment of the function which performs configuration will use all of the  pointers and variables established for the best fit block which were determined during the scan of the linked list. In this fashion the team will not need to dig too deep into the inner workings of the code and can use the efficient algorithms already employed by the first fit slob.   
  \normalfont \indent In order to perform the best fit algorithm across the entire set of allocated page structures the team determined that the \textit{slob\_alloc()} would require some editing. The first fit algorithms approach is straight forward. A loop is entered in which the doubly linked list containing the allocated page structures is scanned until a page with sufficient free space is located. This page is then taken and the \textit{slob\_page\_alloc()} function is called which slots in the memory allocation request in the first space with sufficient size. To implement the best-fit algorithm across all available pages the team decided to write a helper function which would scan the current page with sufficient size very similar to \textit{slob\_page\_allow()} and return the difference in size between the best fit slob size and the size of the memory allocation request. This number can then be compared to other numbers from other pages to determine the overall best page in which the request should be serviced. The team has a strong intuition that this implementation of the best fit algorithm will be substantially slower than the original first fit algorithm, though it will conserve resources to a far greater degree.  

\section{\bf Questions}
        \textbf{What do you think the main point of this assignment is?}\\
                \normalfont \indent The purpose of this assignment was to learn about how memory is allocated at the kernel level. Usually programs use virtual memory provided by the kernel to allocate objects and store information.  With this assignment we looked at the different types of algorithms, and their various shortfalls.  The SLOB allocator that we were comparing against was pretty simple to grasp, making it easier grasp how memory allocation and management work. Additionally, we had to understand and write algorithms that would result in fitting requests into the optimal memory location. \\\\
       \textbf{How did you personally approach the problem? Design decisions, algorithm, etc.}\\
	  		\normalfont \indent Our approach to completing this assignment was to first do research. The team had to get up to speed on Linux SLAB allocators and their theory of operation. Once the team had a good conceptual understanding of how the slob allocator was configured and a rough idea of the programs structure the team analyzed the first fit method. This was the most time consuming part of the assignment as the team had to trace through many lines of kernel code to figure out how the program worked. Understanding the first fit algorithm laid the foundation for successfully writing the best fit algorithm. After a rough prototype for the best fit algorithm was completed the team tested the algorithm only on the current page. This took some debugging. This kernel assignment required substantially more lines of code than the previous assignments and thus there huge room for small mistakes. It took the team some time to get the VM booted and running properly and with the best fit algorithm operating properly over a single page the team tweaked the \textit{slob\_alloc()} function to perform best fit over every allocated page. With this complete the team set about configuring system calls and writing testing scripts for the grading TA's.
 
	\textbf{How did you ensure your solution was correct?}\\
                \normalfont \indent To ensure our solution was correct we ran a program in the background of our VM which allocates random blocks of memory. While this program was running in the background we ran a demo script which triggers the system calls that print the total amount of free space and the total amount of consumed space. When executing this testing method against both the first fit and best fit slob algorithm implementations we noticed a much higher ratio between the amount of free and consumed space on the best fit algorithm. This is precisely what one would expect as the best fit algorithm conserved space in each page by locating the specific block among all pages in which the memory allocation request best fits. Thus the best fit algorithm drastically reduces the amount of internal fragmentation. 
  
        \textbf{What did you learn?}\\
        		\normalfont \indent The team learned a substantial amount about the inner working of the Linux memory management systems. We learned why SLAB allocators improve performance, and some fundamental concepts regarding data fragmentation (both external and internal) and how fragmentation can be avoided. Additionally the team learned how to configure system calls. In conclusion this assignment offered a huge amount of learning opportunities and was both challenging and fun. 

        \textbf{How should the TA evaluate your work?}\\
	\section{\bf Work Log}

                %\begin{tabular}{l l l l}\textbf{Link} & \textbf{Date} & \textbf{Author} & \textbf{Description}\\\hline
\href{https://github.com/ebrahimk/CS444/commit/e10cf2a5be3cb7528d156eb991bb4ee94c5ac0f3}{e10cf2a} & Tue Apr 10 22:42:32 2018 -0700 & ebrahimk & added Makefile and tex template\\\hline
\href{https://github.com/ebrahimk/CS444/commit/399064e4a3c386c1aed3ebf025f2e964411049fb}{399064e} & Tue Apr 10 22:45:33 2018 -0700 & ebrahimk & made project1 folder\\\hline
\href{https://github.com/ebrahimk/CS444/commit/7e41f8e334d9b67756aa7fb715773a4360e83a3b}{7e41f8e} & Tue Apr 10 22:51:18 2018 -0700 & ebrahimk & Added another tex template\\\hline
\href{https://github.com/ebrahimk/CS444/commit/90bed413ed27774ee7f3d84703ea90d9f32e4484}{90bed41} & Wed Apr 11 13:41:22 2018 -0700 & cascadeth & Added scripts for easy setup of the VM for hw1\\\hline
\href{https://github.com/ebrahimk/CS444/commit/225209cf1bb27caaab8d148d66ad2cebcc347828}{225209c} & Wed Apr 11 17:35:33 2018 -0700 & ebrahimk & Progress on tex report\\\hline
\href{https://github.com/ebrahimk/CS444/commit/a22de85663bb053c8335c5e3592ea39b65c292a0}{a22de85} & Wed Apr 11 17:43:53 2018 -0700 & ebrahimk & IEEEtran styling added\\\hline
\href{https://github.com/ebrahimk/CS444/commit/466bf6d68d7bb5055f687bad89014f47d1a98575}{466bf6d} & Wed Apr 11 20:22:08 2018 -0700 & ebrahimk & Git log added for reports\\\hline
\href{https://github.com/ebrahimk/CS444/commit/bfdf3b3f6ce5e223fe99ec5bf17eb73c64853601}{bfdf3b3} & Wed Apr 11 20:34:19 2018 -0700 & ebrahimk & remove log\\\hline
\href{https://github.com/ebrahimk/CS444/commit/604b920fe6c569ed6a59f81d457f5c07590b2661}{604b920} & Thu Apr 12 15:22:05 2018 -0700 & cascadeth & Updated script with colors and clarifications; updated readme with clarifications\\\hline
\href{https://github.com/ebrahimk/CS444/commit/aed5444563d69dd80a0e0dc05e9659e82afc980a}{aed5444} & Thu Apr 12 16:24:00 2018 -0700 & cascadeth & updated gitignore\\\hline
\href{https://github.com/ebrahimk/CS444/commit/c18877d67d024ff8a33b14138f21d754cd89d805}{c18877d} & Thu Apr 12 16:32:36 2018 -0700 & cascadeth & removed .DS_Store files; now included in gitignore\\\hline
\href{https://github.com/ebrahimk/CS444/commit/7ee3a3b2feefb1a785d704251be9c1867c7bd3cb}{7ee3a3b} & Thu Apr 12 17:17:11 2018 -0700 & ebrahimk & Added Gitlog script\\\hline
\href{https://github.com/ebrahimk/CS444/commit/603e40262677cec416eb147b7e31a03c597cfce1}{603e402} & Thu Apr 12 18:45:58 2018 -0700 & bl3rg & First concurrency assignment, should be complete\\\hline
\href{https://github.com/ebrahimk/CS444/commit/7d5d6439601af7d72474fcf711d44191da20b207}{7d5d643} & Thu Apr 12 18:59:48 2018 -0700 & bl3rg & Completed first concurrency assignment\\\hline
\href{https://github.com/ebrahimk/CS444/commit/b6b9eb246d8520fe1435c3c33019cc1dbc70b537}{b6b9eb2} & Mon Apr 16 21:37:03 2018 -0700 & ebrahimk & Reorganized files according to assignment-1\\\hline
\href{https://github.com/ebrahimk/CS444/commit/82ecafd78c700dbc91097ddf33fdac2f1f6d9419}{82ecafd} & Fri May 4 13:14:17 2018 -0700 & Kamron & added my github name\\\hline
\href{https://github.com/ebrahimk/CS444/commit/dff1676205e257abd96e343da4013e50ae01e203}{dff1676} & Fri May 4 13:18:30 2018 -0700 & Kamron & adding project1 file updates\\\hline
\href{https://github.com/ebrahimk/CS444/commit/d5b2af8a3f44f2fba74528489a524bba680763b2}{d5b2af8} & Fri May 4 13:20:52 2018 -0700 & Kamron & assignment 2\\\hline
\href{https://github.com/ebrahimk/CS444/commit/f2110844dc7d778d3f6d29a7a0e235f2ca69d4df}{f211084} & Fri May 4 13:44:34 2018 -0700 & Kamron & Merge branch 'master' of https://github.com/ebrahimk/CS444\\\hline
\href{https://github.com/ebrahimk/CS444/commit/11ee94cb0b8000f4317934ec3ab14919fc8e8495}{11ee94c} & Fri May 4 13:49:32 2018 -0700 & Kamron &  reorganizing\\\hline
\href{https://github.com/ebrahimk/CS444/commit/f274fb3d1cd919b019742c322e3ea8d98f55cca7}{f274fb3} & Fri May 4 13:50:51 2018 -0700 & Kamron & reorganizing\\\hline
\href{https://github.com/ebrahimk/CS444/commit/ee2720ac8cdb00dd6e09203976606ff0e98010b6}{ee2720a} & Fri May 4 13:51:48 2018 -0700 & Kamron & reorganizing\\\hline
\href{https://github.com/ebrahimk/CS444/commit/86650ed58df77c2dc69b1aed453878ec360b6526}{86650ed} & Fri May 4 14:36:56 2018 -0700 & Kamron & added python test script modified qemu script\\\hline
\href{https://github.com/ebrahimk/CS444/commit/68bb5f47a651687c5de0a882a70d1f35a0064165}{68bb5f4} & Fri May 4 17:57:10 2018 -0700 & Kamron & generated python script to grab sector data for plot\\\hline
\href{https://github.com/ebrahimk/CS444/commit/048e41943ebf06341dcfd826548f5ed68a435277}{048e419} & Fri May 4 23:51:38 2018 -0700 & Kamron & adding look implementation\\\hline
\href{https://github.com/ebrahimk/CS444/commit/c80f0e6bb8a253f2bc8328b9fe55efcfe095da42}{c80f0e6} & Sat May 5 01:48:06 2018 -0700 & Kamron & updated scheduler\\\hline
\href{https://github.com/ebrahimk/CS444/commit/a231bfbe9e4e5321231b75e5e036dccd86509853}{a231bfb} & Sat May 5 14:08:16 2018 -0700 & Kamron & added merging functionality to CLOOK algorithm as well as a merge testing python script\\\hline\end{tabular}


\newpage


\end{singlespace}
\restoregeometry
\end{document}

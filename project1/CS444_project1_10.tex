\documentclass[10pt,onecolumn,draftclsnofoot]{IEEEtran} %may need comma after
\usepackage{times}
\usepackage{graphicx}
\usepackage{array}
\usepackage{float}
\usepackage{geometry}
\usepackage{titling}
\usepackage{longtable,hyperref}
\newcommand{\longtableendfoot}
\newgeometry{left=1.905cm, right=1.905cm}
%this is a comment
\title{ Project One, CS444, Spring 2018, Homework Group 10}
\author{Kamron Ebrahimi, Maximillian Schmidt \& Brendan Byers \\ ebrahimk, schmidtm, byersb }
\date{\today}

\begin{document}
\begin{titlingpage}
\maketitle
\begin{abstract}
This report details the process of building the Linux Kernel and running the kernel on the Oregon State University, operating systems II server. To accomplish this task, the command line based, virtual machine, QEMU, was cloned and configured in order to produce a safe, experimental enviorment in which the Linux Kernel can be modified. This paper will walk the reader through the key components of configuring this enviroment.     %paper abstract here
\end{abstract}
\end{titlingpage}


\tableofcontents

\newpage
\section{\bf  Commands Used}

  \paragraph{\normalfont \indent In order to run the Linux Kernel in a QEMU virtual machine a folder titled "10" was created with the following file path /scratch/spring2018/10. A copy of the linux-yocto repository was then cloned and unzipeed to the group 10 folder using the following commands from the group 10 folder.
  }

  \begin{itemize}
    \item \textbf{wget http://git.yoctoproject.org/cgit.cgi/linux-yocto/snapshot/linux-yocto-3.19.2.zip }
    \item \textbf{unzip -q linux-yocto-3.19.2.zip}
  \end{itemize}

  \paragraph{\normalfont With the linux-yocto repository successfully cloned the QEMU enviroment needed to be set up and run. Some preliminary measures were required to ensure this was performed properly. First the a script located at the file path /scratch/opt/environment-setup-i586-poky-linux was sourced. All members of group ten operate in the bash shell, thus this script at the aforementioned file path was used. To source this script the following command was executed from the root directory of os2.engr.oregonstate.edu.}

  \begin{itemize}
    \item \textbf{source /scratch/opt/environment-setup-i586-poky-linux}
  \end{itemize}

  \paragraph{\normalfont In order to ensure that the Makefile associated with the Linux Kernel executed properly the file located at the file path /scratch/files/config-3.19.2-yocto-standard was copied to a newly created folder named ".config" located within the linux-yocto-3.19.2 folder. Additionally the starting Kernel image, bzImage-qemux86.bin and drive file core-image-lsb-sdk-qemux86.ext4 were copied within the linux-yocto-3.19.2 folder. To accomplish this the following commands were executed from within the linux-yocto-3.19.2 folder.
  }

  \begin{itemize}
    \item \textbf{cp /scratch/files/config-3.19.2-yocto-standard ./.config}
    \item \textbf{cp /scratch/files/bzImage-qemux86.bin .}
    \item \textbf{cp /scratch/files/core-image-lsb-sdk-qemux86.ext4 .}
  \end{itemize}

  \paragraph{\normalfont With the aforementioned preliminary configurations complete, the clone of the Linux Kernel could be comiled. The following command allocates four threads to build the Linux Kernel. A limit of the number of threads is used to ensure that no one group consumes to much computational resources on the shared os2.engr.oregonstate.edu server.}

  \begin{itemize}
    \item \textbf{make -j4 all}
  \end{itemize}

  \paragraph{\normalfont With the Kernel built, the last step was to boot the Kernel from within an instance of the QEMU virtual machine. First an instance of QEMU was executed on port 5510 of the os2 server using the following command:}

  \begin{itemize}
    \item \textbf{qemu-system-i386 -gdb tcp::5510 -S -nographic -kernel bzImage-qemux86.bin -drive file=core-image-lsb-sdk-qemux86.ext4,if=virtio -enable-kvm -net none -usb -localtime --no-reboot --append "root=/dev/vda rw console=ttyS0 debug"}
  \end{itemize}

  \paragraph{\normalfont Using GDB from another terminal window the Kernel could be executed. To complete this a group member had to connect via port 5510 via GDB to the QEMU virtual machine and execute the "vmlinux" process from within the linux-yocto-3.19.2 folder. The following chain of commands were used while an instance QEMU was running on port 5510, these commands were executed from within the group 10 folder.
  }

  \begin{itemize}
    \item \textbf{\$GDB linux-yocto-3.14/vmlinux}
    \item \textbf{target remote :5510}
    \item \textbf{continue}
  \end{itemize}

  \paragraph{\normalfont The above commands create a gdb connection to port 5510 of the os2 server and enters the vmlinux file. The vmlinux executable when run hoists an instance of the Linux Kernel from within the QEMU virtual machine. By running the "continue" command from within the gdb connection, vmlinux is executed which boots the Kernel from within QEMU.
  }

\section{\bf  QEMU Command Explanation}

  \begin{itemize}
    \item \textbf{-gdb tcp::5510}
    \begin{itemize}
      \item This will cause QEMU to wait for a connection on a device, which in this case is a tcp port. The connection can also be UDP, pseudo TTY, or standard output. Using standard output allows you to start QEMU from within gdb and establish the connection with a pipe.
    \end{itemize}

    \item \textbf{-S}
    \begin{itemize}
      \item This signals to qemu to not start the CPU at startup.
    \end{itemize}

    \item \textbf{-nographic}
    \begin{itemize}
      \item This will start qemu with graphical output completely disabled. The emulated serial port will usually be redirected to the console.
    \end{itemize}

    \item \textbf{-kernel bzImage-qemux86.bin}
    \begin{itemize}
      \item  Points to the kernel image. This image can either be a standard linux kernel or a kernel in multiboot format.
    \end{itemize}

    \item \textbf{-drive file=core-image-lsb-sdk-qemux86.ext4,if=virtio -enable-kvm}
    \begin{itemize}
      \item The “-drive” defines a new drive, which includes creating the block driver and the guest device. Our drive in this case is “”.  Next, it checks if virtio is enabled and enables the kvm accordingly. Virtio is a virtualization standard that allows the guest to know that is is in a virtual environment.
    \end{itemize}

    \item \textbf{-net none}
    \begin{itemize}
      \item This is used to create an onboard Network Interface Card. In our case we don’t want to create any network interfaces for our kernel. If this option isn’t specified, then a single NIC is created.
    \end{itemize}

    \item \textbf{-usb}
    \begin{itemize}
      \item This flag will enable the usb driver for the guest vm.
    \end{itemize}

    \item \textbf{-localtime }
    \begin{itemize}
      \item This is used to specify that the vm should use local time. This flag has been depreciated as of qemu 2.12.0 and was replaced by the -rtc flag.
    \end{itemize}

    \item \textbf{--no-reboot}
    \begin{itemize}
      \item This will stop qemu from exiting when the guest shuts down. Instead, it will only stop the emulation.
    \end{itemize}

    \item \textbf{--append "root=/dev/vda rw console=ttyS0 debug"}
    \begin{itemize}
      \item This will tell the emulator to use the command line in quotes to be used as the kernel command line. This line here dictates the location of the root directory, what sort of console, and to boot in debug mode.
    \end{itemize}

  \end{itemize}

\section{\bf Git Log}

  %% This file was generated by the script latex-git-log
%% Base git commit URL: https://github.com/ypid/typesetting/commit
\begin{tabular}{lp{12cm}}
  \label{tabular:legend:git-log}
  \textbf{acronym} & \textbf{meaning} \\
  V & \texttt{version} \\
  tag & \texttt{git tag} \\
  MF & Number of \texttt{modified files}. \\
  AL & Number of \texttt{added lines}. \\
  DL & Number of \texttt{deleted lines}. \\
\end{tabular}

\bigskip

\iflanguage{ngerman}{\shorthandoff{"}}{}
\begin{longtable}{|rllp{13cm}rrr|}
\hline \multicolumn{1}{|c}{\textbf{V}} & \multicolumn{1}{c}{\textbf{tag}}
& \multicolumn{1}{c}{\textbf{date}}
& \multicolumn{1}{c}{\textbf{commit message}} & \multicolumn{1}{c}{\textbf{MF}}
& \multicolumn{1}{c}{\textbf{AL}} & \multicolumn{1}{c|}{\textbf{DL}} \\ \hline
\endhead

\hline \multicolumn{7}{|r|}{\longtableendfoot} \\ \hline
\endfoot

\hline% \hline
\endlastfoot

\hline 1 &  & 2012-08-18 & \href{https://github.com/ypid/typesetting/commit/3ff663fcfbc56662426740cf6bab2a840e320db6}{added template.dtx from http://texhacks.blogspot.de/2011/01/simpler-dtx-template.html} & 2 & 127 & 0 \\
\hline 2 &  & 2012-08-19 & \href{https://github.com/ypid/typesetting/commit/e9fc9a2847fe14067bb3ec8d4e4298ff8806f11f}{made my first changes on the docstrip template} & 2 & 40 & 48 \\
\hline 3 &  & 2012-08-19 & \href{https://github.com/ypid/typesetting/commit/cbc7bbcdda54f7b43f2039531d2a4f58c35f8964}{optimized} & 1 & 5 & 4 \\
\hline 4 &  & 2012-08-22 & \href{https://github.com/ypid/typesetting/commit/55d994c391c509897dd56a6a755c51a1e62bd4d4}{optimized} & 2 & 119 & 115 \\
\hline 5 &  & 2012-08-23 & \href{https://github.com/ypid/typesetting/commit/933466b223d70adeda6a865b4b13b4cdfa245070}{added my primary template} & 6 & 252 & 6 \\
\hline 6 &  & 2012-08-23 & \href{https://github.com/ypid/typesetting/commit/5995d70f82b680d3a529aa592558dcdde6f407b7}{added PDFs} & 2 & 0 & 0 \\
\hline 7 &  & 2012-08-23 & \href{https://github.com/ypid/typesetting/commit/7e1efb265b43fb9aad3fa94b22408951076016f3}{added README (Credits)} & 1 & 1 & 0 \\
\hline 8 &  & 2012-08-26 & \href{https://github.com/ypid/typesetting/commit/619a7f6db915bc1e05d0c9ee374aaff53195b257}{converted template.dtx and template.ins to a template for sty2dtx} & 7 & 49 & 62 \\
\hline 9 &  & 2012-08-27 & \href{https://github.com/ypid/typesetting/commit/06f948149a3b0bd2c81062078add19af6bb1ac5d}{optimized} & 6 & 48 & 22 \\
\hline 10 &  & 2012-08-27 & \href{https://github.com/ypid/typesetting/commit/09316b528dc90f45fa4ef6cf9cf5886c4e25c5bb}{added scripts} & 6 & 222 & 0 \\
\hline 11 &  & 2012-08-27 & \href{https://github.com/ypid/typesetting/commit/4c434dcd45aa4a73112b1393d096776eaa938b83}{added README for scripts} & 1 & 10 & 0 \\
\hline 12 &  & 2012-08-27 & \href{https://github.com/ypid/typesetting/commit/794a3db5bd4f0165a347f0628c78f32350731b2a}{optimized scripts} & 4 & 9 & 6 \\
\hline 13 &  & 2012-08-28 & \href{https://github.com/ypid/typesetting/commit/7c80a541828c3e7896c7f55c8b56d745b84a5414}{added MyPackages} & 22 & 1162 & 2 \\
\hline 14 &  & 2012-08-28 & \href{https://github.com/ypid/typesetting/commit/b408bc9fa5f91ff5a7687fc4903d520ab09a288d}{added my Makefile} & 5 & 106 & 0 \\
\hline 15 &  & 2012-08-28 & \href{https://github.com/ypid/typesetting/commit/d86bd28e3f0d0e14d7eb239eb6ad9afab2da47c0}{added more scripts} & 4 & 1465 & 0 \\
\hline 16 &  & 2012-08-28 & \href{https://github.com/ypid/typesetting/commit/ab0554ce8d44852a76b7ad0246e04268b8347acb}{optimized} & 5 & 4 & 1459 \\
\hline 17 &  & 2012-08-31 & \href{https://github.com/ypid/typesetting/commit/75a5b9d653b2798a9cb8f10535f50abf98e94cd2}{optimized} & 7 & 52 & 31 \\
\hline 18 &  & 2012-09-09 & \href{https://github.com/ypid/typesetting/commit/2a99a061ae89f5bd6bd6ace4aac66ad9a768b0fc}{optimized} & 4 & 23 & 10 \\
\hline 19 &  & 2012-09-16 & \href{https://github.com/ypid/typesetting/commit/c2463e8229fbc506a70b4efb157679ecf3aae795}{optimized} & 6 & 10 & 7 \\
\hline 20 &  & 2012-09-23 & \href{https://github.com/ypid/typesetting/commit/1b6cb46337c516e3042d94956e7bf4f097822a47}{added more packages} & 9 & 491 & 12 \\
\hline 21 &  & 2012-09-23 & \href{https://github.com/ypid/typesetting/commit/92a41f208a275b6fcdfe9c442be8c135ec0d1c92}{added school templates} & 25 & 502 & 25 \\
\hline 22 &  & 2012-09-30 & \href{https://github.com/ypid/typesetting/commit/32e21641dff2ffd9fecfe33b93def53d6a305963}{added more templates} & 20 & 537 & 4 \\
\hline 23 &  & 2012-11-03 & \href{https://github.com/ypid/typesetting/commit/2309fa9dd41d0acaf74d6de66976f7ae6f1e7c6b}{rewrote LaTeX-git-log as perl script} & 2 & 63 & 61 \\
\hline 24 &  & 2012-11-03 & \href{https://github.com/ypid/typesetting/commit/773e5b801cfd2ad9a2af8c9d379dd0e2f9afd3cf}{optimized} & 2 & 5 & 4 \\
\hline 25 &  & 2012-11-28 & \href{https://github.com/ypid/typesetting/commit/4ffc0488959ada2a97ba50d5a7db1390afce4a26}{added more templates and optimized} & 12 & 141 & 3 \\
\hline 26 &  & 2013-04-04 & \href{https://github.com/ypid/typesetting/commit/dc3b091e5e8d68daa1c4b73ab96e7bc201a6ed08}{Updated templates.} & 17 & 92 & 55 \\
\hline 27 &  & 2013-04-04 & \href{https://github.com/ypid/typesetting/commit/457664094f33618c62348fe5a9798cc98117bcad}{Added macro for showing the github URL and optimized.} & 8 & 26 & 38 \\
\hline 28 &  & 2013-04-04 & \href{https://github.com/ypid/typesetting/commit/71547c919a2b36548e65320be8ce0d318b473523}{Optimized} & 3 & 36 & 0 \\
\hline 29 &  & 2013-04-20 & \href{https://github.com/ypid/typesetting/commit/3681cf7bc6d18878ca03af4009bdfc9e8835dbcd}{Optimized.} & 5 & 40 & 12 \\
\hline 30 &  & 2013-04-20 & \href{https://github.com/ypid/typesetting/commit/c0ef81a414620007d6f39fa95e59888045fa5cf0}{Optimized.} & 12 & 67 & 22 \\
\hline 31 &  & 2013-04-21 & \href{https://github.com/ypid/typesetting/commit/782b0e0e51773b02cf909600444da892c890b926}{Optimized.} & 4 & 30 & 13 \\
\hline 32 &  & 2013-05-22 & \href{https://github.com/ypid/typesetting/commit/f3048459117aa782b32324bd08ce77bfde325a62}{Optimized (mainly LaTeX-git-wdiff).} & 9 & 229 & 78 \\
\hline 33 &  & 2013-06-15 & \href{https://github.com/ypid/typesetting/commit/c5c45122518ad0e6cf9712dc248b26fd3eea4cf0}{Optimized.} & 3 & 15 & 12 \\
\hline 34 & v1.0 & 2013-06-15 & \href{https://github.com/ypid/typesetting/commit/af3d63d88a8e37c45c4dc7c8b54ee55091f7ea80}{Made latex-git-log ready for CTAN.} & 6 & 482 & 199 \\
\hline 35 &  & 2013-06-15 & \href{https://github.com/ypid/typesetting/commit/489927feca92be10ddb5bde1b8fb11dae6b37bbb}{Optimized latex-git-log.} & 6 & 12 & 15 \\
\hline 36 &  & 2013-06-16 & \href{https://github.com/ypid/typesetting/commit/6d0ff0a6ae2baba715704f7dea23ebb0ac9ddc02}{Optimized.} & 3 & 106 & 79 \\
\hline 37 &  & 2013-06-16 & \href{https://github.com/ypid/typesetting/commit/ff9d61b343971ca964ca5e2708253abe3dad4dab}{Fixed layout.} & 1 & 9 & 5 \\
\hline 38 &  & 2013-06-16 & \href{https://github.com/ypid/typesetting/commit/2c128bf3eb4d734a11f3e7ebfdc93c01eafbd6f4}{Taking care about special charters in the commit message.} & 2 & 48 & 10 \\
\hline 39 &  & 2013-06-16 & \href{https://github.com/ypid/typesetting/commit/7b1683a671fe687ca58f895564341ee4c82a7f7d}{Fixed problem with escaped $\backslash$href.} & 1 & 10 & 3 \\
\hline 40 &  & 2013-06-16 & \href{https://github.com/ypid/typesetting/commit/6b25666f78ff276bbe161a0a7b659ad0c836aa22}{Made the script prettier with perltidy.} & 1 & 33 & 25 \\
\hline 41 &  & 2013-06-16 & \href{https://github.com/ypid/typesetting/commit/38698ff91b2253abddacd135e332d9ace5127122}{Also escape the author field.} & 3 & 10 & 10 \\
\end{longtable}


  \begin{verbatim}
  	latex-git-log --width=13 --lang=en > example-output.tex
  \end{verbatim}
  
\restoregeometry
\end{document}
